\section{Mô hình hóa sơ đồ Use Case}
\subsection{Danh sách Actor}
Sau khi mô tả được hệ thống và xác định được các vai trò, đối tượng cũng như yêu cầu của hệ thống, thì hệ thống sẽ bao gồm các actor như sau:
\begin{itemize}
    \item \hi{Nhân viên bãi xe (Attendant):} thao tác trên UI, hỗ trợ phát vé, thu phí, xác minh mất vé, đăng ký vé tháng.
    \item \hi{Quản trị viên (Admin):} quản lý cấu hình bảng giá, quản lý hạ tầng bãi xe, tài khoản, xem báo cáo.
    \item \hi{Cổng thanh toán ngoài (External Payment Gateway):} xử lý thanh toán online.
    \item \hi{Thiết bị cổng \& Camera (GateDevice):} đọc thẻ/biển số, in vé, mở barrier.
\end{itemize}
\subsection{Sơ đồ use case}
\begin{figure}[H]
        \centering
        \includegraphics[width=1.1\linewidth]{Pic/OOP SPS 20251 - Use case (1).png}
        \caption{Sơ đồ use case của hệ thống}
        \label{fig:placeholder}
    \end{figure}

\subsection{Đặc tả các use case}
\subsubsection*{1. Tiếp nhận xe vào bãi}

\begin{longtable}{|L{4cm}|L{11cm}|}
\hline
\rowcolor[HTML]{FFC0CB}
\textbf{Nội dung} & \textbf{Mô tả} \\
\hline
\endfirsthead

\hline
\rowcolor[HTML]{FFC0CB}
\textbf{Nội dung} & \textbf{Mô tả} \\
\hline
\endhead

\hline
\endfoot

% ================= UC01 =================
\textbf{Tên use case} & Tiếp nhận xe vào bãi \\
\hline

\textbf{Chức năng} &
Ghi nhận lượt gửi xe khi phương tiện vào bãi bằng thẻ từ và biển số; tạo phiên gửi xe đang hoạt động. \\
\hline

\textbf{Tác nhân} &
Nhân viên bãi xe; Thiết bị cổng \& camera; \\
\hline

\textbf{Sự kiện kích hoạt} &
Phương tiện đến cổng vào và nhân viên quẹt thẻ từ. \\
\hline

\textbf{Tiền điều kiện} &
(1) Hệ thống hoạt động và kết nối thiết bị đọc thẻ và đọc biển số. \newline
(2) Thẻ từ hợp lệ. \newline 
(3) Nếu là vé tháng: vé còn hiệu lực và được gắn với xe. \\
\hline

\textbf{Hậu điều kiện} &
(1) Tạo bản ghi lượt gửi xe với thời gian vào và biển số. \newline
(2) Trạng thái lượt gửi = \texttt{Đang gửi}. 
(3) Lưu log thao tác. \\
\hline

\textbf{Luồng cơ bản} &
1. Khách/cư dân đưa xe đến cổng vào và nhân viên bãi xe quẹt thẻ check-in. \newline
2. Hệ thống đọc thẻ, xác định loại thẻ. \newline
3. Hệ thống đọc và ghi nhận biển số lúc vào. \newline
4. Hệ thống tạo lượt gửi xe và lưu dữ liệu. \newline
5. Cho phép xe vào bãi (mở barie). \\
\hline

\textbf{Luồng thay thế} &
A1. Thẻ không hợp lệ $\rightarrow$ từ chối cho vào. \newline 
A2. Vé tháng hết hạn $\rightarrow$ xử lý theo lượt hoặc từ chối. \newline
A3. Không đọc được biển số $\rightarrow$ quét lại. \newline
A4. Thẻ đang có lượt \texttt{Đang gửi} $\rightarrow$ từ chối check-in. \\
\hline
\end{longtable}

\subsubsection*{2. Cho xe ra khỏi bãi}
\begin{longtable}{|L{4cm}|L{11cm}|}
\hline
\rowcolor[HTML]{FFC0CB}
\textbf{Nội dung} & \textbf{Mô tả} \\
\hline
\endfirsthead

\hline
\rowcolor[HTML]{FFC0CB}
\textbf{Nội dung} & \textbf{Mô tả} \\
\hline
\endhead

\hline
\endfoot
\textbf{Tên use case} & Cho xe ra bãi \& tính phí \\
\hline

\textbf{Chức năng} &
Kết thúc lượt gửi xe, đối chiếu thẻ và biển số, tính phí theo bảng giá hoặc miễn phí theo vé tháng. \\
\hline

\textbf{Tác nhân} &
Nhân viên bãi xe; Thiết bị cổng \& camera. \\
\hline

\textbf{Sự kiện kích hoạt} &
Phương tiện đến cổng ra và nhân viên quẹt thẻ từ. \\
\hline

\textbf{Tiền điều kiện} &
(1) Tồn tại lượt gửi xe đang hoạt động tương ứng với thẻ. \newline
(2) Hệ thống kết nối thiết bị đọc thẻ/biển số. \\
\hline

\textbf{Hậu điều kiện} &
(1) Cập nhật thời gian ra và cập nhật trạng thái thẻ về rỗng để dùng cho lần sau. \newline 
(2) Xác định số tiền cần thanh toán (nếu có). \newline
(3) Tạo yêu cầu thanh toán nếu gửi theo lượt. \\
\hline

\textbf{Luồng cơ bản} &
1. Quẹt thẻ tại cổng ra. \newline
2. Truy xuất lượt gửi đang hoạt động. \newline
3. Đọc và đối chiếu biển số. \newline
4. Nếu vé tháng hợp lệ $\rightarrow$ miễn phí. \newline 
5. Nếu gửi theo lượt $\rightarrow$ tính phí. \newline
6. Yêu cầu thanh toán và xử lý. \newline
7. Cho xe ra bãi. \\
\hline

\textbf{Luồng thay thế} &
B1. Biển số không khớp $\rightarrow$ không cho xe ra hoặc làm thủ tục mất vé. \newline
B2. Không đọc được biển số $\rightarrow$ quẹt lại. \newline
B3. Không tìm thấy lượt gửi $\rightarrow$ xử lý thành mất thẻ. \\
\hline

\end{longtable}

\subsubsection*{3. Xử lý thanh toán}
\begin{longtable}{|L{4cm}|L{11cm}|}
\hline
\rowcolor[HTML]{FFC0CB}
\textbf{Nội dung} & \textbf{Mô tả} \\
\hline
\endfirsthead

\hline
\rowcolor[HTML]{FFC0CB}
\textbf{Nội dung} & \textbf{Mô tả} \\
\hline
\endhead

\hline
\endfoot

\textbf{Tên use case} & Thanh toán phí gửi xe \\
\hline

\textbf{Chức năng} &
Thực hiện thanh toán phí gửi xe qua cổng thanh toán bên ngoài, hệ thống không nhận tiền mặt. \\
\hline

\textbf{Tác nhân} &
Cổng thanh toán bên ngoài. \\
\hline

\textbf{Sự kiện kích hoạt} &
Phát sinh phí cần thanh toán khi lấy xe ra hoặc khi đăng ký/gia hạn vé tháng. \\
\hline

\textbf{Tiền điều kiện} &
(1) Có số tiền cần thanh toán và mã giao dịch. \newline
(2) Kết nối được với cổng thanh toán. \\
\hline

\textbf{Hậu điều kiện} &
(1) Thanh toán thành công $\rightarrow$ ghi nhận giao dịch, cho phép hoàn tất dịch vụ.  \newline
(2) Thất bại/hủy $\rightarrow$ không cho xe ra hoặc không kích hoạt dịch vụ. \\
\hline

\textbf{Luồng cơ bản} &
1. Hiển thị số tiền và tạo yêu cầu thanh toán.  \newline
2. Khách thanh toán qua cổng thanh toán.  \newline
3. Cổng trả kết quả.  \newline
4. Nếu thành công $\rightarrow$ cập nhật trạng thái nghiệp vụ. \\
\hline

\textbf{Luồng thay thế} &
C1. Không thể thanh toán online $\rightarrow$ nhân viên thanh toán hộ.  \newline
C2. Timeout/thất bại $\rightarrow$ cho phép thử lại.  \newline
C3. Mất kết nối cổng thanh toán $\rightarrow$ ghi nhận lỗi. \\
\hline

\hline

\end{longtable}
\subsubsection*{4. Đăng ký vé tháng mới}
\begin{longtable}{|L{4cm}|L{11cm}|}
\hline
\rowcolor[HTML]{FFC0CB}
\textbf{Nội dung} & \textbf{Mô tả} \\
\hline
\endfirsthead

\hline
\rowcolor[HTML]{FFC0CB}
\textbf{Nội dung} & \textbf{Mô tả} \\
\hline
\endhead

\hline
\endfoot
% ================= UC04 =================
\textbf{Tên use case} & Đăng ký vé tháng \\
\hline

\textbf{Chức năng} &
Tạo mới vé tháng, liên kết thẻ với người đăng ký và phương tiện, thiết lập thời hạn sử dụng. \\
\hline

\textbf{Tác nhân} &
Quản trị viên hoặc Nhân viên bãi xe; Cổng thanh toán bên ngoài. \\
\hline

\textbf{Sự kiện kích hoạt} &
Có yêu cầu đăng ký vé tháng. \\
\hline

\textbf{Tiền điều kiện} &
(1) Phương tiện chưa gắn với vé tháng khác (nếu ràng buộc 1-1). \\
\hline

\textbf{Hậu điều kiện} &
(1) Vé tháng được tạo và ở trạng thái \texttt{Có hiệu lực} sau khi thanh toán.  \newline
(2) Liên kết vé tháng với thẻ và thông tin xe/biển số. \\
\hline

\textbf{Luồng cơ bản} &
1. Chọn chức năng đăng ký vé tháng.  \newline
2. Nhập thông tin người đăng ký, xe, thời hạn vé.  \newline
3. Hệ thống kiểm tra dữ liệu và tính phí.  \newline
4. Tạo yêu cầu thanh toán.  \newline
5. Thanh toán thành công $\rightarrow$ kích hoạt vé tháng. \\
\hline

\textbf{Luồng thay thế} &
D1. Biển số đã tồn tại trong vé khác $\rightarrow$ cảnh báo.  \newline
D2. Thanh toán thất bại $\rightarrow$ không tạo hoặc tạo vé ở trạng thái chờ. \\
\hline

\end{longtable}
\subsubsection*{5. Gia hạn vé tháng}
\begin{longtable}{|L{4cm}|L{11cm}|}
\hline
\rowcolor[HTML]{FFC0CB}
\textbf{Nội dung} & \textbf{Mô tả} \\
\hline
\textbf{Tên use case} & Gia hạn vé tháng \\
\hline
\textbf{Chức năng} & Gia hạn thời hạn sử dụng của vé tháng hiện có sau khi thanh toán thành công. \\
\hline
\textbf{Tác nhân} & Quản trị viên hoặc Nhân viên bãi xe; Cổng thanh toán bên ngoài; \\
\hline
\textbf{Sự kiện kích hoạt} & Người đăng ký yêu cầu gia hạn vé tháng\\
\hline
\textbf{Tiền điều kiện} & 
(1) Vé tháng tồn tại. \\
& (2) Vé tháng chưa bị hủy. \\
\hline
\textbf{Hậu điều kiện} & 
(1) Cập nhật thời hạn mới của vé tháng. \\
& (2) Ghi nhận giao dịch gia hạn. \\
\hline
\textbf{Luồng cơ bản} &
1. Nhân viên chọn chức năng gia hạn vé tháng. \\
& 2. Nhập/tra cứu vé tháng theo thẻ hoặc mã vé. \\
& 3. Hệ thống kiểm tra trạng thái vé và tính phí gia hạn theo gói. \\
& 4. Hệ thống tạo yêu cầu thanh toán. \\
& 5. Sau khi thanh toán thành công, hệ thống cập nhật ngày hết hạn mới và lưu lịch sử gia hạn. \\
\hline
\textbf{Luồng thay thế} &
E1. Vé tháng đã bị hủy $\rightarrow$ không cho gia hạn, yêu cầu đăng ký mới. \\
& E2. Vé tháng không tồn tại $\rightarrow$ thông báo lỗi và không cho gia hạn. \\
& E3. Thanh toán thất bại $\rightarrow$ thanh toán lại hoặc không cập nhật thời hạn. \\
\hline
\end{longtable}

\subsubsection*{6. Xử lý mất vé xe}

\begin{longtable}{|L{4cm}|L{11cm}|}
\hline
\rowcolor[HTML]{FFC0CB}
\textbf{Nội dung} & \textbf{Mô tả} \\
\hline
\textbf{Tên use case} & Xử lý mất vé \\
\hline
\textbf{Chức năng} &
Xử lý trường hợp người gửi xe bị mất thẻ từ; xác minh thông tin và cho phép xe ra bãi theo quy định sau khi hoàn tất nghĩa vụ thanh toán phát sinh. \\
\hline
\textbf{Tác nhân} &
Nhân viên bãi xe; Cổng thanh toán bên ngoài. \\
\hline
\textbf{Sự kiện kích hoạt} &
Người gửi xe thông báo mất thẻ tại thời điểm muốn cho xe ra. \\
\hline
\textbf{Tiền điều kiện} &
(1) Nhân viên có quyền xử lý mất thẻ. \\
& (2) Có thông tin để xác minh phương tiện (biển số, thời gian vào, dữ liệu camera). \\
\hline
\textbf{Hậu điều kiện} &
(1) Lượt gửi được kết thúc theo quy trình mất thẻ. \\
& (2) Ghi nhận phí xử lý mất thẻ và phí gửi xe. \\
& (3) Có log sự kiện mất thẻ. \\
\hline
\textbf{Luồng cơ bản} &
1. Nhân viên chọn chức năng xử lý mất thẻ. \\
& 2. Nhập thông tin xác minh (biển số, loại xe, thời gian vào). \\
& 3. Hệ thống tìm kiếm lượt gửi phù hợp và hiển thị kết quả. \\
& 4. Nhân viên kiểm tra, xác nhận lượt gửi đúng. \\
& 5. Hệ thống tính phí gửi xe và phí xử lý mất thẻ theo quy định. \\
& 6. Hệ thống tạo yêu cầu thanh toán. \\
& 7. Sau khi thanh toán thành công, hệ thống kết thúc lượt gửi và cho phép xe ra bãi. \\
\hline
\textbf{Luồng thay thế}&
F1. Không tìm thấy lượt gửi phù hợp $\rightarrow$ chuyển xử lý thủ công theo quy định quản lý. \\
& F2. Thông tin không đủ để xác minh $\rightarrow$ từ chối cho xe ra hoặc yêu cầu bổ sung giấy tờ. \\
& F3. Thanh toán thất bại $\rightarrow$ thanh toán lại hoặc không cho xe ra. \\
\hline
\end{longtable}
