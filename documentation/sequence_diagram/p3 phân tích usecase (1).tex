\section{Phân tích use case}
\subsection{UC tiếp nhận xe vào }
\subsubsection*{1. Sơ đồ tuần tự của UC tiếp nhận xe vào}

\begin{figure}[H]
    \centering
    \includegraphics[width=1\linewidth]{Pic/check in.drawio.png}
    \caption{Sơ đồ tuần tự của UC tiếp nhận xe vào}
    \label{fig:placeholder}
\end{figure}

\subsubsection*{2. Suy ra các class kĩ thuật từ sơ đồ tuần tự}
\begin{longtable}{|L{1.5cm}|L{2.8cm}|L{4.3cm}|L{6.5cm}|}
\hline
\rowcolor[HTML]{FFC0CB}
\textbf{Loại} & \textbf{Tên class} & \textbf{Vai trò trong sequence Check-in} & \textbf{Properties \& Methods tối thiểu cần có} \\
\hline
\endfirsthead

\hline
\rowcolor[HTML]{FFC0CB}
\textbf{Loại} & \textbf{Tên class} & \textbf{Vai trò trong sequence Check-in} & \textbf{Properties \& Methods tối thiểu cần có} \\
\hline
\endhead

\hline
\endfoot

% --- ROWS ---
actor &
\textbf{Attendant} &
Người thao tác check-in: chọn check-in, quẹt vé tháng hoặc vé lượt & \\
\hline

boundary &
\textbf{EntryUI} &
Màn hình thao tác check-in: gửi yêu cầu check-in tới controller; hiển thị thông báo ``Bãi đầy'', ``Mời vào'', hiển thị vé &
\textbf{Methods:} \texttt{requestCheckIn(gateId)}; \texttt{showFullMessage()}; \texttt{showTicketInfo(ticketInfo)} ; \texttt{showError(message)}. \newline
\textbf{Properties:} \texttt{gateId}, \texttt{vehicleInfo}. \\
\hline

boundary (device) &
\textbf{GateDevice} &
Thiết bị cổng: đọc thẻ từ, mở cổng, đọc biển số xe, ghi nhận thời gian vào &
\textbf{Methods:} \texttt{readCardID(): String}; \texttt{issueHourlyTicket(): String}; \texttt{openGate(gateId): void} \newline
\textbf{Properties:} \texttt{deviceId}, \texttt{gateId}. \\
\hline

\multirow{2}{*}{controller} &
\textbf{CheckIn Controller} &
Điều phối nghiệp vụ check-in: kiểm tra bãi có đầy không; rẽ nhánh vé tháng/vé lượt; tạo vehicle/ticket/session; gọi mở cổng &
\textbf{Methods:} \texttt{requestCheckIn(gateId)}; \texttt{checkInByCard(cardId, gateId)}; \texttt{checkInWithTicket(ticketId, vehicle, gateId)}; \texttt{isFull(gateId): bool} (gửi sang \textbf{ParkingLot}). \newline
\textbf{Properties:} tham chiếu tới \texttt{MembershipController}, \texttt{ParkingLot}. \\
\cline{2-4}

& \textbf{Membership Controller} &
Xác minh vé tháng và trả về thông tin membership + vehicle &
\textbf{Methods:} \texttt{verifyMembership(cardId): MembershipInfo}. \\
\hline

\multirow{5}{*}{entity} &
\textbf{ParkingLot} &
Nghiệp vụ bãi xe: kiểm tra đầy/chỗ trống theo gate; tìm zone phù hợp cho xe &
\textbf{Methods:} \texttt{isFull(gateId): bool}; \texttt{findZone(vehicle, gateId): ParkingZone}. \newline
\textbf{Properties:} \texttt{zones: List<ParkingZone>}. \\
\cline{2-4}

& \textbf{Vehicle} &
Thông tin xe, được tạo/cập nhật từ TicketID hoặc lấy từ membership &
\textbf{Methods:} \texttt{createOrUpdateVehicle(ticketId): Vehicle}. \\
\cline{2-4}

& \textbf{Ticket} &
Vé lượt phát khi không có vé tháng &
\textbf{Properties:} \texttt{ticketId}, \texttt{issueTime}, \texttt{gateId}. \\
\cline{2-4}

& \textbf{ParkingSession} &
Phiên gửi xe được tạo khi check-in; vé lượt gắn vehicle + ticket + zone, trả về session &
\textbf{Methods:} \texttt{createSession(vehicle, ticket, zone): ParkingSession} (lên ParkingSession lifeline). \newline
\textbf{Properties:} \texttt{sessionId}, \texttt{entryTime}, \texttt{vehicle}, \texttt{ticket}, \texttt{zone}, \texttt{status}. \\
\cline{2-4}

 & \textbf{ParkingZone} &
Khu/zone nhận xe, được ParkingLot trả về để tạo phiên gửi &
\textbf{Properties:} \texttt{zoneId}, \texttt{capacity}, \texttt{activeSessionsCount}. \\
\hline

entity/ \newline DTO &
\textbf{Membership Info} &
Kết quả xác minh vé tháng &
\textbf{Properties tối thiểu:} \texttt{isValid}; \texttt{vehicle}; \texttt{ticketId/plan/endDate}. \\
\hline

\end{longtable}

\subsubsection*{3. Mối quan hệ giữa các class}

\begin{itemize}
  \item \textbf{EntryUI $\rightarrow$ CheckInController}: EntryUI gửi yêu cầu check-in đến CheckInController. \\
  \textit{Dependency}  

  \item \textbf{CheckInController $\rightarrow$ GateDevice}: Controller sử dụng GateDevice để đọc thẻ, phát vé và mở cổng.  \\
  \textit{Association (uses)} 

  \item \textbf{CheckInController $\rightarrow$ MembershipController}: Kiểm tra xe có vé tháng hay không. \\
  \textit{Dependency} 

  \item \textbf{CheckInController $\rightarrow$ ParkingLot}: Kiểm tra bãi có đầy không và tìm zone phù hợp. \\
  \textit{Association}
  
  \item \textbf{ParkingLot $\rightarrow$ ParkingZone}:  Một ParkingLot quản lý nhiều ParkingZone. \\
  \textit{Aggregation} 

  \item \textbf{ParkingZone $\rightarrow$ ParkingSession}:   Tạo ParkingSession khi xe vào bãi. \\
  \textit{Creates} 

  \item \textbf{ParkingSession $\rightarrow$ Vehicle}:   Mỗi session gắn với một phương tiện. \\
  \textit{Association}

  \item \textbf{ParkingSession $\rightarrow$ Ticket}:  Vé lượt được gắn với session (nếu không có vé tháng). \\
  \textit{Association} 
\end{itemize}


\subsection{UC cho xe ra }
\subsubsection*{1. Sơ đồ tuần tự của UC cho xe ra}
\begin{figure}[H]
    \centering
    \includegraphics[width=0.8\linewidth]{Pic/check out.drawio.png}
    \caption{Sơ đồ tuần tự của UC cho xe ra}
    \label{fig:placeholder}
\end{figure}

\subsubsection*{2. Suy ra các class kĩ thuật từ sơ đồ tuần tự}
% =========================
% TABLE: CHECK-IN (UC01)
% =========================
\begin{longtable}{|L{1.5cm}|L{2.8cm}|L{4.3cm}|L{6.5cm}|}
\hline
\rowcolor[HTML]{FFC0CB}
\textbf{Loại} & \textbf{Tên class} & \textbf{Vai trò trong sequence Check-out} & \textbf{Properties \& Methods tối thiểu cần có} \\
\hline
\endfirsthead

\hline
\rowcolor[HTML]{FFC0CB}
\textbf{Loại} & \textbf{Tên class} & \textbf{Vai trò trong sequence Check-out} & \textbf{Properties \& Methods tối thiểu cần có} \\
\hline
\endhead

% ---- actor ----
\multirow{1}{*}{actor} &
\textbf{Attendant} &
Thao tác quẹt vé / xác nhận thông tin / xác nhận thanh toán & \\
\hline

% ---- boundary (external) ----
\multirow{2}{*}{boundary} &
\textbf{Payment Gateway} &
Xử lý giao dịch thanh toán &
\texttt{requestPayment(amount, paymentInfo): PaymentResult} \\
\hline

% ---- boundary ----
\multirow{2}{*}{boundary} &
\textbf{ExitUI} &
Giao diện check-out: nhận thao tác quẹt vé, hiển thị thông báo, form mất vé, phí, kết quả thanh toán &
\texttt{loadOpenSessions()}; \texttt{showOpenSessions(openSessions)}; \texttt{selectSession(selectedSessionId)}; \texttt{submitQR(qrData)}; \texttt{readCardOrPlate()}; \texttt{verifyMembership(cardOrPlateId)}; \texttt{printLostTicketForm()}; \texttt{showFeeInfo(feeAmount)}; \texttt{showMessage(message)}; \\
\cline{2-4}
& \textbf{GateDevice} &
Đọc thẻ/biển số và mở cổng &
\textbf{Properties:} \texttt{gateId, status} \newline \texttt{openGate(gateId)} \\
\hline

% ---- controller ----
\multirow{3}{*}{controller} &
\textbf{CheckOut Controller} &
Kiểm vé tháng, tìm session, xử lý vé lượt/mất vé, tính phí, gọi thanh toán, mở cổng &
\texttt{loadOpenSessions()}; \texttt{selectSession(selectedSessionId)};
\texttt{submitQR(qrData)}; \texttt{parsePlate(qrData): plate} \\
\cline{2-4}
& \textbf{Membership Controller} &
Kiểm tra xe có vé tháng hay không &
\texttt{verifyMembershipByCardOrPlate (cardOrPlateId): MembershipInfo} \\
\cline{2-4}
& \textbf{Payment Controller} &
Thực hiện nghiệp vụ thanh toán cho session (vé lượt hoặc mất vé) &
\texttt{processPayment(sessionToPay, feeAmount, paymentInfo): PaymentResult}; \\
\hline

% ---- entity ----
\multirow{3}{*}{entity} &
\textbf{ParkingSession} &
Phiên gửi xe cần thanh toán hoặc cho xe ra &
\texttt{listOpenSessions(): List<ParkingSession>};
\texttt{getById(id): ParkingSession};
\texttt{attachPayment(payment)};
\texttt{setExitTime(time)};
\texttt{close()};
\texttt{createLostTicketSession (vehicleInfo, feeAmount): ParkingSession}. \\
\cline{2-4}
& \textbf{PricePolicy} &
Tính phí gửi xe và phí mất vé &
\texttt{calculateFee(session): double}; \texttt{calculateLostTicketFee (vehicleInfo):double};\\
& & &
\texttt{calculateParkingFeeForLostTicket (vehicleInfo, now): double} \\
\cline{2-4}
& \textbf{Payment} &
Đại diện giao dịch thanh toán được tạo và hoàn tất &
\texttt{createPayment(amount)}; \texttt{markedCompleted()} \\
\hline

% ---- entity/DTO + ngầm dùng ----
\multirow{1}{*}{entity } &
\textbf{Membership Info} &
Kết quả kiểm tra vé tháng (true/false) &
\texttt{hasMonthlyTicket: boolean, validTo?, memberId?} \\ 
(DTO) & & \\ \hline
\end{longtable}

\subsubsection*{3. Mối quan hệ giữa các class}
\begin{itemize}
  \item \textbf{ExitUI $\rightarrow$ CheckOutController}: Gửi yêu cầu check-out. \\
  \textit{Dependency} 

  \item \textbf{ExitUI $\rightarrow$ PaymentController}: Gửi yêu cầu tạo mã QR để thanh toán. \\
  \textit{Dependency} 

  \item \textbf{CheckOutController $\rightarrow$ GateDevice:}   Dùng để mở cổng. \\
  \textit{Dependency} 


  \item \textbf{CheckOutController $\rightarrow$ MembershipController}:  Kiểm tra vé tháng.\\
  \textit{Dependency}

  \item \textbf{CheckOutController $\rightarrow$ ParkingSession}: Truy xuất và cập nhật session hiện tại. \\
  \textit{Dependency} 
  
  \item \textbf{CheckOutController $\rightarrow$ PricePolicy}: Tính phí gửi xe hoặc phí mất vé. \\
  \textit{Dependency} 
  
  \item \textbf{CheckOutController $\rightarrow$ PaymentController:} Điều phối thanh toán. \\
  \textit{Dependency}   

  \item \textbf{PaymentController $\rightarrow$ Payment:}   Tạo đối tượng Payment khi thanh toán thành công. \\
  \textit{Creates} 
  
  \item \textbf{PaymentController $\rightarrow$ PaymentGateway:} Gọi cổng thanh toán bên ngoài. \\
  \textit{Dependency} 
  
  \item \textbf{ParkingSession $\rightarrow$ Payment:} Session có thể gắn với một Payment.  \\
  \textit{Association} 
\end{itemize}


\subsection{UC đăng ký vé tháng }
\subsubsection*{1. Sơ đồ tuần tự của UC đăng ký vé tháng}
\begin{figure}[H]
    \centering
    \includegraphics[width=1\linewidth]{Pic/sequence đki vé tháng.png}
    \caption{Sơ đồ tuần tự của UC đăng ký vé tháng}
    \label{fig:placeholder}
\end{figure}
\subsubsection*{2. Suy ra các class kĩ thuật từ sơ đồ tuần tự}
\begin{longtable}{|L{1.5cm}|L{2.8cm}|L{4.3cm}|L{6.5cm}|}
\hline
\rowcolor[HTML]{FFC0CB}
\textbf{Loại} & \textbf{Tên class} & \textbf{Vai trò trong sequence đăng ký vé} & \textbf{Properties \& Methods tối thiểu cần có} \\
\hline
\endfirsthead

\hline
\rowcolor[HTML]{FFC0CB}
\textbf{Loại} & \textbf{Tên class} & \textbf{Vai trò trong sequence đăng ký vé} & \textbf{Properties \& Methods tối thiểu cần có} \\
\hline
\endhead

% ================= ACTOR =================
\multirow{1}{*}{actor}
& Attendant
& Thao tác đăng ký vé tháng, nhập thông tin, xác nhận thanh toán & \\
\hline

% ================= BOUNDARY =================
\multirow{1}{*}{boundary}
& MembershipUI
& Giao diện đăng ký vé tháng, nhập thông tin khách \& xe, hiển thị kết quả
& \textbf{Properties:} currentCustomerInfo, currentVehicleInfo, selectedPlan \newline
  \textbf{Methods:} registerMonthlyTicket(customerInfo, vehicleInfo, plan); \newline
  showMembershipInfo(ticketInfo); showError(message)
\\
\hline

% ================= CONTROLLER =================
\multirow{2}{*}{controller}
& Membership Controller
& Điều phối nghiệp vụ đăng ký vé tháng
& \textbf{Properties:} customerRepo, monthlyTicketRepo, policy \newline
  \textbf{Methods:} registerMonthly Ticket(customerInfo, vehicleInfo, plan); \newline
  findOrCreateCustomer(customerInfo); createOrUpdate(vehicleInfo); \newline
  hasActiveTicket(vehicle); createMonthlyTicket(customer, vehicle, plan)
\\
\cline{2-4}

& Payment Controller
& Xử lý thanh toán phí đăng ký vé tháng
& \textbf{Properties:} paymentGateway \newline
  \textbf{Methods:} processPaymentForMembership(customer, amount); \newline
  paymentCompleted()
\\
\hline

% ================= ENTITY =================
\multirow{5}{*}{entity}
& Customers
& Lưu thông tin khách hàng đăng ký vé
& \textbf{Properties:} customerId, name, phone \\
\cline{2-4}

& Vehicles
& Đại diện xe đăng ký vé tháng
& \textbf{Properties:} vehicleId, licensePlate, vehicleType \newline
  \textbf{Methods:} createOrUpdate()
\\
\cline{2-4}

& MonthlyTicket
& Vé tháng được tạo sau khi thanh toán thành công
& \textbf{Properties:} ticketId, startDate, expiryDate, status \newline
  \textbf{Methods:} createMonthlyTicket()
\\
\cline{2-4}

& PricePolicy
& Tính phí đăng ký vé tháng theo gói
& \textbf{Properties:} policyId, name \newline
  \textbf{Methods:} createMembershipFee(plan, vehicleType): double
\\
\cline{2-4}

& Payments
& Thực thể lưu giao dịch thanh toán
& \textbf{Properties:} paymentId, amount, method, status \newline
  \textbf{Methods:} createPayment(amount); markCompleted()
\\
\hline

% ================= EXTERNAL =================
\multirow{1}{*}{external \newline boundary}
& Cổng thanh toán
& Xử lý giao dịch thanh toán bên thứ ba
& \textbf{Methods:} requestPayment(amount, paymentInfo)
\\
\hline

\end{longtable}

\subsubsection*{3. Mối quan hệ giữa các class}
\begin{itemize}
  \item \textbf{MembershipUI $\rightarrow$ MembershipController:} Gửi yêu cầu đăng ký vé tháng. \\
  \textit{Dependency}   

  \item \textbf{MembershipController $\rightarrow$ Customer:} Liên kết vé tháng với khách hàng. \\
  \textit{Association}   

  \item \textbf{MembershipController $\rightarrow$ Vehicle:}   Vé tháng gắn với một phương tiện. \\
  \textit{Association}


  \item \textbf{MembershipController $\rightarrow$ MonthlyTicket:} Tạo MonthlyTicket mới khi đăng ký thành công. \\
  \textit{Creates} 
  
  \item \textbf{MembershipController $\rightarrow$ MembershipPolicy:} Tính phí đăng ký vé tháng. \\
  \textit{Dependency}   

  \item \textbf{MembershipController $\rightarrow$ PaymentController:} Thực hiện thanh toán. \\
  \textit{Dependency}  

  \item \textbf{PaymentController $\rightarrow$ Payment:}  Lưu giao dịch thanh toán. \\
  \textit{Creates}  
\end{itemize}


\subsection{UC gia hạn vé tháng}
\subsubsection*{1. Sơ đồ tuần tự của UC gia hạn vé tháng}
\begin{figure}[H]
    \centering
    \includegraphics[width=1\linewidth]{Pic/sequence gia hạn vé.drawio.png}
    \caption{Sơ đồ tuần tự của UC gia hạn vé tháng}
    \label{fig:placeholder}
\end{figure}

\subsubsection*{2. Suy ra các class kĩ thuật từ sơ đồ tuần tự}
\begin{longtable}{|L{1.5cm}|L{2.8cm}|L{4.3cm}|L{6.5cm}|}
\hline
\rowcolor[HTML]{FFC0CB}
\textbf{Loại} & \textbf{Tên class} & \textbf{Vai trò trong sequence Gia hạn vé} & \textbf{Properties \& Methods tối thiểu cần có} \\
\hline
\endfirsthead

\hline
\rowcolor[HTML]{FFC0CB}
\textbf{Loại} & \textbf{Tên class} & \textbf{Vai trò trong sequence Gia hạn vé} & \textbf{Properties \& Methods tối thiểu cần có} \\
\hline
\endhead

% ================= ACTOR =================
\multirow{1}{*}{actor}
& Attendant
& Chọn chức năng gia hạn vé, xác nhận thanh toán
& \\ \hline

% ================= BOUNDARY =================
\multirow{1}{*}{boundary}
& MembershipUI
& Giao diện gia hạn vé tháng, nhập số tháng, hiển thị phí và kết quả
& \textbf{Properties:} selectedTicketId, selectedMonths, currentFeeAmount \newline
  \textbf{Methods:} renewMonthlyTicket(ticketId, months); \newline
  showRenewInfo(feeAmount, months); \newline
  showError(message); \newline
  showSuccess(message, newEndDate)
\\
\hline

% ================= CONTROLLER =================
\multirow{2}{*}{controller}
& Membership Controller
& Điều phối nghiệp vụ gia hạn vé tháng
& \textbf{Properties:} monthlyTicketRepo, membershipPolicy \newline
  \textbf{Methods:} renewMonthlyTicket(ticketId, months); \newline
  findById(ticketId); \newline
  isExpired(now); \newline
  extendEndDate(months, now)
\\
\cline{2-4}

& Payment Controller
& Xử lý thanh toán phí gia hạn vé tháng
& \textbf{Properties:} paymentGateway \newline
  \textbf{Methods:} processPaymentFor Membership(ticketId, feeAmount, method, paymentInfo); \newline
  paymentCompleted()
\\
\hline

% ================= ENTITY =================
\multirow{5}{*}{entity}
& MonthlyTicket
& Vé tháng được kiểm tra trạng thái và gia hạn thời gian
& \textbf{Properties:} ticketId, startDate, expiryDate, status \newline
  \textbf{Methods:} isExpired(now): boolean; \newline
  extendEndDate(months, now)
\\
\cline{2-4}

& Membership Policy
& Tính phí gia hạn vé tháng theo gói
& \textbf{Properties:} policyId, name \newline
  \textbf{Methods:} calculateMembershipFee(plan, vehicleType): double
\\
\cline{2-4}

& Payments
& Lưu giao dịch thanh toán gia hạn vé
& \textbf{Properties:} paymentId, amount, method, status \newline
  \textbf{Methods:} createPayment(amount); markedCompleted()
\\
\cline{2-4}

& Vehicles
& Xe gắn với vé tháng
& \textbf{Properties:} vehicleId, vehicleType, licensePlate \\ \hline

% ================= EXTERNAL BOUNDARY =================
\multirow{1}{*}{external \newline boundary}
& Cổng thanh toán ngoài
& Xử lý giao dịch thanh toán bên thứ ba
& \textbf{Methods:} requestPayment(amount, paymentInfo)
\\
\hline

\end{longtable}

\subsubsection*{3. Mối quan hệ giữa các class}
\begin{itemize}
  \item \textbf{MembershipUI $\rightarrow$ MembershipController:} Gửi yêu cầu gia hạn vé. \\
  \textit{Dependency}   

  \item \textbf{MembershipController $\rightarrow$ MonthlyTicket:}  Truy xuất và cập nhật vé tháng. \\
  \textit{Association} 

  \item \textbf{MembershipController $\rightarrow$ MembershipPolicy:}  Tính phí gia hạn. \\
  \textit{Dependency} 


  \item \textbf{MembershipController $\rightarrow$ PaymentController:}  Điều phối thanh toán gia hạn. \\
  \textit{Dependency} 

  \item \textbf{PaymentController $\rightarrow$ Payment:}   Tạo Payment cho giao dịch gia hạn. \\
  \textit{Creates} 
  
  \item \textbf{MonthlyTicket $\rightarrow$ Vehicle:}   Vé tháng gắn với phương tiện để xác định loại phí. \\
  \textit{Association} 
\end{itemize}

\subsection{UC hủy vé tháng}
\subsubsection*{1. Sơ đồ tuần tự của UC hủy vé tháng}
\begin{figure}[H]
    \centering
    \includegraphics[width=1\linewidth]{Pic/sequence hủy vé.drawio.png}
    \caption{Sơ đồ tuần tự của UC hủy vé tháng}
    \label{fig:placeholder}
\end{figure}

\subsubsection*{2. Suy ra các class kĩ thuật từ sơ đồ tuần tự}
\begin{longtable}{|L{1.5cm}|L{2.8cm}|L{4.3cm}|L{6.5cm}|}
\hline
\rowcolor[HTML]{FFC0CB}
\textbf{Loại} & \textbf{Tên class} & \textbf{Vai trò trong sequence Hủy vé} & \textbf{Properties \& Methods tối thiểu cần có} \\
\hline
\endfirsthead

\hline
\rowcolor[HTML]{FFC0CB}
\textbf{Loại} & \textbf{Tên class} & \textbf{Vai trò trong sequence Hủy vé} & \textbf{Properties \& Methods tối thiểu cần có} \\
\hline
\endhead

% ================= ACTOR =================
\multirow{1}{*}{actor}
& Admin
& Chọn chức năng huỷ vé tháng & \\
\hline

% ================= BOUNDARY =================
\multirow{1}{*}{boundary}
& MembershipUI
& Giao diện cho admin thao tác huỷ vé tháng và hiển thị kết quả
& \textbf{Properties:} selectedTicketId, lastMessage \newline
  \textbf{Methods:} cancelMonthlyTicket(ticketId); \newline
  showError(message); \newline
  showSuccess(message, status)
\\
\hline

% ================= CONTROLLER =================
\multirow{1}{*}{controller}
& Membership Controller
& Điều phối nghiệp vụ huỷ vé tháng
& \textbf{Properties:} monthlyTicketRepo \newline
  \textbf{Methods:} cancelMonthlyTicket(ticketId); \newline
  findById(ticketId); \newline
  expiredDurationMonths(now): int
\\
\hline

% ================= ENTITY =================
\multirow{3}{*}{entity}
& MonthlyTicket
& Vé tháng bị kiểm tra thời gian hết hạn và cập nhật trạng thái
& \textbf{Properties:} ticketId, startDate, expiryDate, status \newline
  \textbf{Methods:} expiredDurationMonths(now): int; \newline
  cancel(now, reason)
\\
\cline{2-4}

& Vehicles
& Xe gắn với vé tháng 
& \textbf{Properties:} vehicleId, licensePlate\\
\hline

\end{longtable}

\subsubsection*{3. Mối quan hệ giữa các class}
\begin{itemize}
  \item \textbf{MembershipUI $\rightarrow$ MembershipController:} Gửi yêu cầu hủy vé tháng. \\
  \textit{Dependency} 
  

  \item \textbf{MembershipController $\rightarrow$ MonthlyTicket:} Truy xuất vé tháng cần hủy. \\
  \textit{Association}   

  \item \textbf{MembershipController $\rightarrow$ MonthlyTicket:} Kiểm tra thời gian hết hạn để quyết định hủy. \\
  \textit{Uses} 

  \item \textbf{MonthlyTicket $\rightarrow$ Vehicle:} Vé tháng liên kết với phương tiện. \\
  \textit{Association}

  \item \textbf{MonthlyTicket $\rightarrow$ Customer:}   Vé tháng thuộc về một khách hàng. \\
  \textit{Association} 
\end{itemize}

\subsection{UC xem báo cáo và thống kê}
\subsubsection*{1. Sơ đồ tuần tự của UC xem báo cáo và thống kê}
\begin{figure}[H]
    \centering
    \includegraphics[width=1\linewidth]{Pic/sequence thống kê.drawio.png}
    \caption{Sơ đồ tuần tự của UC xem báo cáo và thống kê}
    \label{fig:placeholder}
\end{figure}

\subsubsection*{2. Suy ra các class kĩ thuật từ sơ đồ tuần tự}
\begin{longtable}{|L{1.5cm}|L{2.8cm}|L{4.3cm}|L{6.5cm}|}
\hline
\rowcolor[HTML]{FFC0CB}
\textbf{Loại} & \textbf{Tên class} & \textbf{Vai trò trong sequence xem báo cáo} & \textbf{Properties \& Methods tối thiểu cần có} \\
\hline
\endfirsthead

\hline
\rowcolor[HTML]{FFC0CB}
\textbf{Loại} & \textbf{Tên class} & \textbf{Vai trò trong sequence Check-in} & \textbf{Properties \& Methods tối thiểu cần có} \\
\hline
\endhead

actor & Admin & Chọn loại báo cáo và khoảng thời gian & \\ \hline

boundary &
ReportUI &
Giao diện nhận input và hiển thị báo cáo &
\textbf{Properties:} startDate, endDate, reportType \newline
\textbf{Methods:} requestReport(type, startDate, endDate); \  renderChart(data); \ renderTable(data) \\
\hline

controller &
ReportController &
Điều phối use case xem báo cáo &
\textbf{Properties:} reportFactory \newline
\textbf{Methods:} requestReport(type, startDate, endDate) \\
\hline

entity (abstract) &
Report &
Lớp trừu tượng đại diện cho một báo cáo &
\textbf{Properties:} startDate, endDate \newline
\textbf{Methods:} generateData(): ReportData\\
\hline

entity &
RevenueReport &
Báo cáo doanh thu, chứa logic tính toán &
\textbf{Properties:} totalRevenue, revenueByDate \newline
\textbf{Methods:} generateData();\ calculateTotal(fee); \ groupByDate() \\
\hline

entity &
ParkingSession &
Nguồn dữ liệu nghiệp vụ để tính doanh thu &
\textbf{Properties:} sessionId, entryTime, exitTime, feeAmount, status \newline
\textbf{Methods:} getFee() \\
\hline

entity / DTO &
ReportData &
Dữ liệu đã xử lý trả về cho UI &
\textbf{Properties:} chartData, tableData, summary \\
\hline
\end{longtable}

\subsubsection*{3. Mối quan hệ giữa các class}
\begin{itemize}
  \item \textbf{ReportUI $\rightarrow$ ReportController:} Gửi yêu cầu xem báo cáo. \\
  \textit{Dependency} 
  

  \item \textbf{ReportController $\rightarrow$ Report (abstract):} Tạo báo cáo theo loại được chọn. \\
  \textit{Creates} 
  
  \item \textbf{Report $\rightarrow$ RevenueReport / TrafficReport / IncidentReport:} Các loại báo cáo kế thừa từ Report.  \\
  \textit{Generalization (Inheritance)} 
  
  \item \textbf{RevenueReport $\rightarrow$ ParkingSession:} Sử dụng dữ liệu session để tính doanh thu.  \\
  \textit{Dependency} 
  
  \item \textbf{TrafficReport $\rightarrow$ ParkingSession:} Thống kê số lượt xe. \\
  \textit{Dependency} 

  \item \textbf{Report $\rightarrow$ ReportData:} Sinh dữ liệu báo cáo để trả về UI. \\
  \textit{Creates} 
  
\end{itemize}

\section{Tái cấu trúc các Controller/Service chính}
% ==== Gợi ý packages cần có ====
% \usepackage{longtable}
% \usepackage[table]{xcolor}
% \usepackage{array}
% \usepackage{multirow}
% \usepackage{colortbl}
% \newcolumntype{L}[1]{>{\raggedright\arraybackslash}p{#1}}

% =========================================================
% 1) CheckOutController — cần chia nhỏ nhất
% =========================================================
\subsection{CheckOutController}

\subsubsection*{Nguyên nhân cần chia nhỏ}
Trong sequence check-out, \texttt{CheckOutController} đang làm quá nhiều việc:
\begin{itemize}
  \item Load danh sách xe đang gửi, chọn session.
  \item Parse QR + đối chiếu biển số.
  \item Xác minh vé tháng.
  \item Xử lý vé lượt/ mất vé + xác minh giấy tờ.
  \item Tính phí (gọi \texttt{PricePolicy} nhiều kiểu).
  \item Gọi \texttt{PaymentController} + quyết định mở cổng.
\end{itemize}


\subsubsection*{Đề xuất chia nhỏ}

\paragraph{A) Use case điều phối}
\begin{longtable}{|L{3.2cm}|L{4.0cm}|L{9.0cm}|}
\hline
\rowcolor[HTML]{FFC0CB}
\textbf{Loại} & \textbf{Tên class} & \textbf{Properties \& Methods tối thiểu cần có} \\
\hline

Application / UseCase &
\texttt{CheckOutUseCase} &
\textbf{Props:} \texttt{sessionRepo}, \texttt{qrValidator}, \texttt{membershipVerifier}, \texttt{feeService}, \texttt{paymentService}, \texttt{gateService}, \texttt{clock}. \newline
\textbf{Methods:} \texttt{loadOpenSessions()}; \texttt{selectSession(id)}; \texttt{submitQr(qrData)}; \texttt{checkout(cardOrPlateId, gateId, paymentInfo, vehicleInfo?)} \\
\hline

\end{longtable}

\paragraph{B) Tách từng trách nhiệm kỹ thuật}

\begin{longtable}{|L{3.2cm}|L{4.0cm}|L{9.0cm}|}
\hline
\rowcolor[HTML]{FFC0CB}
\textbf{Loại} & \textbf{Tên class} & \textbf{Properties \& Methods tối thiểu cần có} \\
\hline
\endfirsthead

\hline
\rowcolor[HTML]{FFC0CB}
\textbf{Loại} & \textbf{Tên class} & \textbf{Properties \& Methods tối thiểu cần có} \\
\hline
\endhead

\multirow{4}{*}{Domain service} &
\texttt{QrPlateValidator} &
\textbf{Methods:} \texttt{parsePlate(qrData): Plate}; \texttt{ensureMatch(plateFromQr, plateSelected)} \\
\cline{2-3}

& \texttt{MembershipVerifier} &
\textbf{Props:} \texttt{monthlyTicketRepo} (hoặc gọi Membership module).\newline
\textbf{Methods:} \texttt{verifyByCardOrPlate(id): MembershipInfo} \\
\cline{2-3}

& \texttt{FeeService} &
\textbf{Props:} \texttt{pricingPolicy}.\newline
\textbf{Methods:} \texttt{calcFeeForSession(session, now): Money}; \texttt{calcFeeForLostTicket(vehicleInfo, now): Money} \\
\cline{2-3}

& \texttt{LostTicketHandler} &
\textbf{Props:} \texttt{vehicleDocValidator}, \texttt{sessionFactory}.\newline
\textbf{Methods:} \texttt{validateDocs(vehicleInfo): bool}; \texttt{buildLostSession(vehicleInfo, feeAmount, now): ParkingSession} \\
\hline

\multirow{2}{*}{Application service} &
\texttt{GateService} &
\textbf{Props:} \texttt{gateDevice}.\newline
\textbf{Methods:} \texttt{openGate(gateId)} \\
\cline{2-3}

& \texttt{PaymentService} &
\textbf{Props:} \texttt{paymentGateway}, \texttt{paymentRepo}.\newline
\textbf{Methods:} \texttt{pay(amount, paymentInfo): Payment}; \texttt{payAndAttach(session, amount, paymentInfo): PaymentResult} \\
\hline

\end{longtable}

\subsubsection*{Mối quan hệ sau khi chia}
\begin{itemize}
  \item \texttt{ExitUI} $\rightarrow$ \texttt{CheckOutController} (thin controller)
  \item \texttt{CheckOutController} $\rightarrow$ \texttt{CheckOutUseCase} (dependency)
  \item \texttt{CheckOutUseCase} $\rightarrow$ (\texttt{QrPlateValidator}, \texttt{MembershipVerifier}, \texttt{FeeService}, \\ \texttt{LostTicketHandler}, \texttt{PaymentService}, \texttt{GateService}, \texttt{ParkingSessionRepository})
  \item \texttt{PaymentService} $\rightarrow$ \texttt{PaymentGateway}, \texttt{PaymentRepository}
  \item \texttt{GateService} $\rightarrow$ \texttt{GateDevice}
\end{itemize}

% =========================================================
% 2) MembershipController — nên tách “quản trị vé tháng” khỏi “nghiệp vụ check-in/out”
% =========================================================
\subsection{MembershipController}
\subsubsection*{Nguyên nhân cần chia nhỏ}
Trong các sequence vé tháng, \texttt{MembershipController} đang được sử dụng ở:
\begin{itemize}
  \item Đăng ký vé: \texttt{findOrCreate Customers}, \texttt{createOrUpdate Vehicles}, kiểm tra active ticket, tính phí membership.
  \item Gia hạn: validate gói tháng 1/3/6, check expired $>$ 12 tháng, tính phí, xử lý thanh toán, \texttt{extendEndDate}.
  \item Hủy: find ticket, tính số tháng expired, rule $>$ 12 tháng mới hủy tự động.
  \item Check-in/out lại gọi \texttt{verifyMembershipByCard/Plate}.
\end{itemize}

$\Rightarrow$ Đây là nhiều bounded context trộn chung: quản lý khách/xe + quản lý vé + rule + payment.

\subsubsection*{Đề xuất chia nhỏ}

\paragraph{A) Module Membership theo UseCase}

\begin{longtable}{|L{3.2cm}|L{4.6cm}|L{8.4cm}|}
\hline
\rowcolor[HTML]{FFC0CB}
\textbf{Loại} & \textbf{Tên class} & \textbf{Properties \& Methods tối thiểu cần có} \\
\hline
\endfirsthead

\hline
\rowcolor[HTML]{FFC0CB}
\textbf{Loại} & \textbf{Tên class} & \textbf{Properties \& Methods tối thiểu cần có} \\
\hline
\endhead

\multirow{3}{*}{UseCase} &
\texttt{RegisterMonthly TicketUseCase} &
\textbf{Props:} \texttt{customerService}, \texttt{vehicleService}, \texttt{monthlyTicketRepo}, \texttt{membershipFeePolicy}, \texttt{paymentService}, \texttt{clock}.\newline
\textbf{Methods:} \texttt{register(customerInfo, vehicleInfo, planMonths, paymentInfo)} \\
\cline{2-3}

& \texttt{RenewMonthly TicketUseCase} &
\textbf{Props:} \texttt{monthlyTicketRepo}, \texttt{membershipPolicy}, \texttt{membershipFeePolicy}, \texttt{paymentService}, \texttt{clock}.\newline
\textbf{Methods:} \texttt{renew(ticketId, months, paymentInfo)} \\
\cline{2-3}

& \texttt{CancelMonthly TicketUseCase} &
\textbf{Props:} \texttt{monthlyTicketRepo}, \texttt{membershipPolicy}, \texttt{clock}.\newline
\textbf{Methods:} \texttt{cancel(ticketId, now)} \\
\hline

\multirow{2}{*}{Domain service} &
\texttt{MembershipPolicy} &
\textbf{Methods:} \texttt{canRenew(ticket, months, now)}; \texttt{canCancel(ticket, now)} (rule ``$>$ 12 months'') \\
\cline{2-3}

& \texttt{MembershipFeePolicy} &
\textbf{Methods:} \texttt{calculate(planMonths, vehicleType): Money} \\
\hline

\multirow{2}{*}{Application service} &
\texttt{CustomerService} &
\textbf{Props:} \texttt{customerRepo}.\newline
\textbf{Methods:} \texttt{findOrCreate(customerInfo): Customer} \\
\cline{2-3}

& \texttt{VehicleService} &
\textbf{Props:} \texttt{vehicleRepo}.\newline
\textbf{Methods:} \texttt{createOrUpdate(vehicleInfo): Vehicle} \\
\hline

\end{longtable}

\paragraph{B) Entity rõ ràng}

\begin{longtable}{|L{3.2cm}|L{4.6cm}|L{8.4cm}|}
\hline
\rowcolor[HTML]{FFC0CB}
\textbf{Loại} & \textbf{Tên class} & \textbf{Properties \& Methods tối thiểu cần có} \\
\hline
\endfirsthead

\hline
\rowcolor[HTML]{FFC0CB}
\textbf{Loại} & \textbf{Tên class} & \textbf{Properties \& Methods tối thiểu cần có} \\
\hline
\endhead

Entity &
\texttt{MonthlyTicket} &
\textbf{Props:} \texttt{id}, \texttt{customerId}, \texttt{vehiclePlate}, \texttt{vehicleType}, \texttt{startDate}, \texttt{endDate}, \texttt{status}.\newline
\textbf{Methods:} \texttt{hasActive(now)}; \texttt{isExpired(now)}; \texttt{expiredDurationMonths(now)}; \texttt{extendEndDate(months, now)}; \texttt{cancel(now, reason)} \\
\hline

\end{longtable}

\subsubsection*{Mối quan hệ sau khi chia}
\begin{itemize}
  \item \texttt{MembershipUI} $\rightarrow$ \texttt{MembershipController}$\rightarrow$ (\texttt{Register/Renew/Cancel UseCase})
  \item \texttt{CheckInUseCase} / \texttt{CheckOutUseCase} $\rightarrow$ \texttt{MembershipVerifier} 
\end{itemize}

% ==== Gợi ý packages cần có ====
% \usepackage{longtable}
% \usepackage[table]{xcolor}
% \usepackage{array}
% \usepackage{multirow}
% \usepackage{colortbl}
% \newcolumntype{L}[1]{>{\raggedright\arraybackslash}p{#1}}

% =========================================================
% 3) PaymentController — tách “orchestration” và “gateway integration”
% =========================================================
\subsection{PaymentController}

\subsubsection*{Nguyên nhân cần chia nhỏ}

Payment xuất hiện ở cả:
\begin{itemize}
  \item checkout vé lượt/mất vé
  \item đăng ký vé tháng
  \item gia hạn vé tháng
\end{itemize}

$\Rightarrow$ Nếu \texttt{PaymentController} làm hết (request gateway + tạo entity + attach session/ticket + retry/fail) sẽ phình cực nhanh.

\subsubsection*{Đề xuất chia nhỏ}

\begin{longtable}{|L{3.0cm}|L{4.2cm}|L{9.0cm}|}
\hline
\rowcolor[HTML]{FFC0CB}
\textbf{Loại} & \textbf{Tên class} & \textbf{Properties \& Methods tối thiểu cần có} \\
\hline
\endfirsthead

\hline
\rowcolor[HTML]{FFC0CB}
\textbf{Loại} & \textbf{Tên class} & \textbf{Properties \& Methods tối thiểu cần có} \\
\hline
\endhead

Application service &
\texttt{PaymentService} &
\textbf{Props:} \texttt{paymentGateway}, \texttt{paymentRepo}. \newline
\textbf{Methods:} \texttt{requestPayment(amount, paymentInfo): PaymentResult}; \texttt{createPayment(amount): Payment}; \texttt{complete(payment, gatewayResult)} \\
\hline

Adapter &
\texttt{PaymentGateway} (interface) &
\textbf{Methods:} \texttt{requestPayment(amount, paymentInfo): PaymentResult} \\
\hline

Entity &
\texttt{Payment} &
\textbf{Props:} \texttt{id}, \texttt{amount}, \texttt{status}, \texttt{method}, \texttt{transactionId?}, \texttt{paidAt?}. \newline
\textbf{Methods:} \texttt{markCompleted(txId, paidAt)}; \texttt{markFailed(reason)} \\
\hline

\end{longtable}

\subsubsection*{Mối quan hệ sau khi chia nhỏ}
\begin{itemize}
  \item \texttt{CheckOutUseCase} $\rightarrow$ \texttt{PaymentService}
  \item \texttt{RegisterMonthlyTicketUseCase} $\rightarrow$ \texttt{PaymentService}
  \item \texttt{RenewMonthlyTicketUseCase} $\rightarrow$ \texttt{PaymentService}
\end{itemize}


% =========================================================
% 4) PricePolicy — tách theo “case tính tiền” để khỏi 1 đống if/else
% =========================================================
\subsection{PricePolicy}

\subsubsection*{Nguyên nhân cần chia nhỏ}

Trong sequence check-out bạn có ít nhất 3 dạng:
\begin{itemize}
  \item tính phí vé lượt: \texttt{calculateFee(session)}
  \item tính phí mất vé: \texttt{calculateLostTicketFee(vehicleInfo)}
  \item tính phí gửi xe khi mất vé: \texttt{calculateParkingFeeForLostTicket(vehicleInfo, now)}
\end{itemize}


\subsubsection*{Đề xuất chia nhỏ}

\begin{longtable}{|L{3.0cm}|L{4.2cm}|L{9.0cm}|}
\hline
\rowcolor[HTML]{FFC0CB}
\textbf{Loại} & \textbf{Tên class} & \textbf{Properties \& Methods tối thiểu cần có} \\
\hline
\endfirsthead

\hline
\rowcolor[HTML]{FFC0CB}
\textbf{Loại} & \textbf{Tên class} & \textbf{Properties \& Methods tối thiểu cần có} \\
\hline
\endhead

Interface &
\texttt{PricingPolicy} &
\textbf{Methods:} \texttt{calculate(input): Money} \\
\hline

\multirow{3}{*}{Strategy} &
\texttt{SessionPricingPolicy} &
\textbf{Methods:} \texttt{calculate(session, now): Money} \\
\cline{2-3}

& \texttt{LostTicketFeePolicy} &
\textbf{Methods:} \texttt{calculate(vehicleType): Money} \\
\cline{2-3}

& \texttt{LostTicket ParkingPolicy} &
\textbf{Methods:} \texttt{calculate(vehicleInfo, now): Money} \\
\hline

Facade &
\texttt{FeeService} &
\textbf{Props:} các policy trên. \newline
\textbf{Methods:} \texttt{calcFeeForSession(...)}; \texttt{calcFeeForLostTicket(...)} \\
\hline

\end{longtable}

\subsubsection*{Mối quan hệ các class sau khi chia nhỏ}
\begin{itemize}
    \item \texttt{ParkingZone} giữ \texttt{pricePolicy} để áp cho zone đó.
    \item \texttt{CheckOutController} gọi đúng policy theo case (vé lượt/mất vé).
\end{itemize}

% =========================================================
% 5) ParkingSession — phải tách Repository/Factory ra khỏi Entity
% =========================================================
\subsection{ParkingSession}

\subsubsection*{Nguyên nhân cần chia nhỏ}

Trong sequence, \textbf{ParkingSession} đang gọi kiểu:
\begin{itemize}
  \item \texttt{listOpenSessions()}
  \item \texttt{getById()}
  \item \texttt{findOpenSession()}
\end{itemize}

\subsubsection*{Đề xuất chia nhỏ}

\begin{longtable}{|L{3.0cm}|L{4.2cm}|L{9.0cm}|}
\hline
\rowcolor[HTML]{FFC0CB}
\textbf{Loại} & \textbf{Tên class} & \textbf{Properties \& Methods tối thiểu cần có} \\
\hline
\endfirsthead

\hline
\rowcolor[HTML]{FFC0CB}
\textbf{Loại} & \textbf{Tên class} & \textbf{Properties \& Methods tối thiểu cần có} \\
\hline
\endhead

Entity &
\texttt{ParkingSession} &
\textbf{Props:} \texttt{id}, \texttt{plate}, \texttt{vehicleType}, \texttt{zoneId}, \texttt{gateIn}, \texttt{gateOut?}, \texttt{entryTime}, \texttt{exitTime?}, \texttt{status}, \texttt{paymentId?}, \texttt{sessionType(NORMAL/LOST/MONTHLY)}. \newline
\textbf{Methods:} \texttt{setExitTime(now)}; \texttt{close()}; \texttt{attachPayment(payment)} \\
\hline

Repository (interface) &
\texttt{ParkingSession Repository} &
\textbf{Methods:} \texttt{listOpenSessions(): List\textless ParkingSession\textgreater}; \texttt{getById(id)}; \texttt{findOpenByPlate(plate)}; \texttt{save(session)} \\
\hline

Factory &
\texttt{ParkingSession Factory} &
\textbf{Methods:} \texttt{createNormal(vehicleInfo, ticket, zoneId, now)}; \texttt{createMonthly(vehicleInfo, zoneId, now)}; \texttt{createLost(vehicleInfo, feeAmount, now)} \\
\hline

\end{longtable}

\subsubsection*{Mối quan hệ}
\begin{itemize}
  \item \texttt{CheckInUseCase} $\rightarrow$ \texttt{ParkingSessionFactory} + \texttt{ParkingSessionRepository}
  \item \texttt{CheckOutUseCase} $\rightarrow$ \texttt{ParkingSessionRepository} (+ Factory khi lost ticket)
\end{itemize}




\subsection{Class Report}
\subsubsection*{Nguyên nhân cần chia nhỏ}
\begin{itemize}
    \item Báo cáo có nhiều loại (Doanh thu, Lưu lượng, Số phương tiện, số vé tháng...), mỗi loại có cách tính và dữ liệu khác nhau.
    \item Nếu dồn hết logic vào \texttt{Report} hoặc \texttt{ReportController} sẽ dẫn đến class phình: khó mở rộng thêm loại report mới, khó test từng loại.
    \item UI báo cáo có nhiều flow: chọn loại $\rightarrow$ nhập khoảng ngày $\rightarrow$ render bảng/biểu đồ \\ $\rightarrow$ export.
\end{itemize}

\subsubsection*{Đề xuất chia nhỏ}

\begin{longtable}{|L{3.0cm}|L{4.2cm}|L{9.0cm}|}
\hline
\rowcolor[HTML]{FFC0CB}
\textbf{Loại} & \textbf{Tên class} & \textbf{Properties \& Methods tối thiểu cần có} \\
\hline
\endfirsthead

\hline
\rowcolor[HTML]{FFC0CB}
\textbf{Loại} & \textbf{Tên class} & \textbf{Properties \& Methods tối thiểu cần có} \\
\hline
\endhead

Entity (abstract base) &
\texttt{Report } &
\textbf{Props:} \texttt{startDate, endDate, generatedDate} \newline
\textbf{Methods:} \texttt{generateData()}; \texttt{getExportData()}.\\
\hline

Entity &
\texttt{RevenueReport} &
\textbf{Props:} \texttt{totalRevenue, revenueByPaymentMethod}\newline
\textbf{Methods:} \texttt{generateData()}\\
\hline

Entity &
\texttt{TrafficReport} &
\textbf{Props:} \texttt{totalVehicles,  vehiclesByType}\newline
\textbf{Methods:} \texttt{generateData()}\\
\hline

Entity &
\texttt{IncidentReport} &
\textbf{Props:} \texttt{incidents}\newline
\textbf{Methods:} \texttt{generateData()}\\
\hline
\end{longtable}

\subsubsection*{Mối quan hệ}

\begin{itemize}
    \item \texttt{ReportController} tạo \texttt{Report}.
\end{itemize}

% --------------------------------------------------

\subsection{Class Vehicle}

\subsubsection*{Nguyên nhân cần chia nhỏ}
\begin{itemize}
    \item Hệ thống có nhiều loại xe, mỗi loại có hệ số tính phí khác nhau.
    \item Nếu dùng 1 class \texttt{Vehicle} chứa hết rule theo \texttt{vehicleType} (if/else) sẽ rối và khó thêm loại mới.
    \item Có biến thể ``xe điện'' (electric-only zone), cần phân biệt rõ hành vi.
\end{itemize}

\subsubsection{Đề xuất chia nhỏ}
% ==== Packages cần có ====
% \usepackage{longtable}
% \usepackage[table]{xcolor}
% \usepackage{array}
% \usepackage{multirow}
% \newcolumntype{L}[1]{>{\raggedright\arraybackslash}p{#1}}

% =========================================================
% Bảng class (3 cột) — Vehicle hierarchy
% =========================================================

\begin{longtable}{|L{3.2cm}|L{4.0cm}|L{8.8cm}|}
\hline
\rowcolor[HTML]{FFC0CB}
\textbf{Loại} & \textbf{Tên class} & \textbf{Properties \& Methods} \\
\hline
\endfirsthead

\hline
\rowcolor[HTML]{FFC0CB}
\textbf{Loại} & \textbf{Tên class} & \textbf{Properties \& Methods} \\
\hline
\endhead

Abstract Entity &
\texttt{Vehicle} &
\textbf{Props:} \texttt{plate: String} . \newline
\textbf{Methods:} \texttt{getFeeFactor(): double} (abstract) \\
\hline

\multirow{5}{*}{Entity} &
\texttt{Car} &
\textbf{Methods:} \texttt{getFeeFactor(): double} \\
\cline{2-3}

& \texttt{ElectricCar} &
\textbf{Methods:} \texttt{getFeeFactor(): double} \\
\cline{2-3}

& \texttt{Motorbike} &
\textbf{Methods:} \texttt{getFeeFactor(): double} \\
\cline{2-3}

& \texttt{ElectricMotorbike} &
\textbf{Methods:} \texttt{getFeeFactor(): double} \\
\cline{2-3}

& \texttt{Bicycle} &
\textbf{Methods:} \texttt{getFeeFactor(): double} \\
\hline

\end{longtable}
\subsubsection*{Mối quan hệ các class}
\begin{itemize}
    \item \texttt{ParkingSession} $\rightarrow$ \texttt{Vehicle} (mỗi session gắn với 1 xe).
    \item \texttt{ParkingLot} / \texttt{ParkingZone} có thể dùng thông tin xe (gas/điện) để chọn zone phù hợp.
\end{itemize}

% =========================================================
% Bảng class (3 cột) — UserAccount hierarchy
% =========================================================






% --------------------------------------------------



\subsection{Class UserAccount}

\subsubsection*{Nguyên nhân cần chia nhỏ}
\begin{itemize}
    \item Hệ thống có vai trò khác nhau (Admin vs Attendant) với quyền khác nhau.
    \item Nếu chỉ dùng 1 \texttt{UserAccount} và check quyền bằng string role sẽ dễ lỗi, rối, khó mở rộng.
    \item Tách lớp giúp thể hiện rõ quyền truy cập UI / use case.
\end{itemize}

\subsubsection*{Đề xuất chia nhỏ}

\begin{longtable}{|L{3.2cm}|L{4.0cm}|L{8.8cm}|}
\hline
\rowcolor[HTML]{FFC0CB}
\textbf{Loại} & \textbf{Tên class} & \textbf{Properties \& Methods} \\
\hline
\endfirsthead

\hline
\rowcolor[HTML]{FFC0CB}
\textbf{Loại} & \textbf{Tên class} & \textbf{Properties \& Methods} \\
\hline
\endhead

Abstract Entity &
\texttt{UserAccount} &
\textbf{Props:} \texttt{userId: String}, \texttt{username: String}, \texttt{passwordHash: String}, \texttt{status: AccountStatus}. \newline
\textbf{Methods (abstract hoặc default):} \texttt{canAccessEntry(): boolean}; \texttt{canAccessExit(): boolean}; \texttt{canAccessReports(): boolean}; \texttt{canManageUsers(): boolean} \\
\hline

\multirow{2}{*}{Entity} &
\texttt{AdminAccount} &
\textbf{Methods:} \texttt{canAccessEntry(): boolean}; \texttt{canAccessExit(): boolean}; \texttt{canAccessReports(): boolean}; \texttt{canManageUsers(): boolean} \\
\cline{2-3}

& \texttt{AttendantAccount} &
\textbf{Methods:} \texttt{canAccessEntry(): boolean}; \texttt{canAccessExit(): boolean}. \\
\hline

\end{longtable}

\subsubsection*{Mối quan hệ các class}
\begin{itemize}
    \item \texttt{UserAccountController}: thao tác CRUD tài khoản qua \texttt{UserRepository}.
    \item UI (Entry / Exit / Report / UserManagement) có thể check quyền trực tiếp từ object account hiện tại.
\end{itemize}

\section{Infrastructure \& Cross-cutting Components}
\subsection{Infrastructure/Transaction}

\subsubsection{Vấn đề}
Các flow kiểu:
\begin{itemize}
  \item \texttt{payment success} $\rightarrow$ tạo \texttt{Payment} $\rightarrow$ attach vào \texttt{ParkingSession} $\rightarrow$ close session $\rightarrow$ mở cổng
  \item \texttt{membership payment success} $\rightarrow$ tạo \texttt{Payment} $\rightarrow$ \texttt{extendEndDate()} $\rightarrow$ update repository
\end{itemize}

Nếu lỗi xảy ra giữa chừng sẽ sinh trạng thái sai (ví dụ Payment đã tạo nhưng Session chưa attach, hoặc vé đã extend nhưng chưa lưu).

\subsubsection{Class cần bổ sung}

\begin{itemize}
  \item \texttt{UnitOfWork} \textless\textless infrastructure\textgreater\textgreater
\end{itemize}

\paragraph{Mục đích}

Đảm bảo các bước cập nhật dữ liệu diễn ra \textbf{atomic} (thành công hết hoặc rollback hết).

\paragraph{Vai trò}

\begin{itemize}
  \item \texttt{runInTx(fn)} hoặc \texttt{begin() / commit() / rollback()}
  \item Gom các thao tác repository trong một transaction logic.
\end{itemize}

\paragraph{Properties / Methods}

\begin{itemize}
  \item Properties: none
  \item Methods: \texttt{runInTx(fn)}, \texttt{begin()}, \texttt{commit()}, \texttt{rollback()}
\end{itemize}

\paragraph{Mối quan hệ}

\begin{itemize}
  \item \texttt{PaymentController} $\rightarrow$ \texttt{UnitOfWork} \textit{(dependency)}
  \item \texttt{MembershipController} $\rightarrow$ \texttt{UnitOfWork} (renew / cancel) \textit{(dependency)}
  \item \texttt{CheckOutController} $\rightarrow$ \texttt{UnitOfWork} \textit{(dependency)}
  \item \texttt{UserAccountController} $\rightarrow$ \texttt{UnitOfWork}
  \item \texttt{SystemScheduler} $\rightarrow$ \texttt{UnitOfWork}
\end{itemize}


% -------------------------------------------------

\subsection{Infrastructure/Time}

\subsubsection{Vấn đề}

Hệ thống dùng \texttt{now} ở nhiều nơi:
\begin{itemize}
  \item \texttt{setExitTime(now)}
  \item \texttt{calculateParkingFeeForLostTicket(vehicleInfo, now)}
  \item \texttt{isExpired(now)}, \texttt{extendEndDate(months, now)}
\end{itemize}

Dùng trực tiếp \texttt{DateTime.now()} khiến code khó test.

\subsubsection{Class cần bổ sung}
\begin{enumerate}
    \item \texttt{TimeProvider} \textless\textless interface\textgreater\textgreater \vspace{-0.5cm}
    \paragraph{Methods}
\begin{itemize}
  \item \texttt{nowDateTime(): DateTime}
  \item \texttt{today(): Date}
\end{itemize}

    \item \texttt{SystemTimeProvider} \textless\textless infrastructure\textgreater\textgreater  \vspace{-0.5cm}
    \paragraph{Methods}

\begin{itemize}
  \item \texttt{nowDateTime(): DateTime}
  \item \texttt{today(): Date}
\end{itemize}
\end{enumerate}

\paragraph{Mối quan hệ}

\begin{itemize}
  \item \texttt{CheckInController, CheckoutController, MembershipController, PaymentController} $\rightarrow$ \texttt{TimeProvider}
  \item \texttt{SystemScheduler} $\rightarrow$ \texttt{TimeProvider}
  \item \texttt{SystemTimeProvider} $\rightarrow$ \texttt{TimeProvider} \textit{(dependency)}
\end{itemize}

% -------------------------------------------------

\subsection{InputValidator}

\subsubsection{Vấn đề}

Rule validate nằm rải rác trong controller sẽ làm controller phình và khó tái sử dụng.

Ví dụ:
\begin{itemize}
  \item renew chỉ cho \{1, 3, 6\} tháng
  \item cancel chỉ khi \texttt{expiredDurationMonths > 12}
  \item lost ticket cần verify ownership
\end{itemize}

\subsubsection{Class cần bổ sung}

\begin{enumerate}
  \item \texttt{MembershipValidator}
\paragraph{Methods}
\begin{itemize}
  \item \texttt{isRenewPlanValid(months: int): bool}
  \item \texttt{canRenew(ticket: MonthlyTicket, now: Date): bool}
  \item \texttt{canCancel(ticket: MonthlyTicket, now: Date): bool}
\end{itemize}  
  \item \texttt{LostTicketValidator}
  \paragraph{Methods}
  \begin{itemize}
      \item \texttt{verifyOwnership(vehicleInfo, ownerDocs): bool} 
  \end{itemize}
\end{enumerate}

\paragraph{Mối quan hệ}

\begin{itemize}
  \item \texttt{MembershipController} $\rightarrow$ \texttt{MembershipValidator}
  \item \texttt{SystemScheduler} $\rightarrow$ \texttt{MembershipValidator}
  \item \texttt{CheckOutController} $\rightarrow$ \texttt{LostTicketValidator}
\end{itemize}

% -------------------------------------------------
\section{Thiết kế lớp cho các chức năng nhỏ}

Bên cạnh các use case chính có luồng xử lý phức tạp đã được phân tích chi tiết bằng sơ đồ tuần tự, hệ thống còn tồn tại một số use case có phạm vi nghiệp vụ hẹp và luồng xử lý tương đối đơn giản. Đối với các use case này, việc xây dựng sơ đồ tuần tự không mang lại nhiều giá trị bổ sung trong việc làm rõ thiết kế. Do đó, dựa trên đặc tả use case và các quy tắc nghiệp vụ đã xác định, có thể suy trực tiếp các lớp tham gia, trách nhiệm của từng lớp cũng như mối quan hệ giữa chúng để xây dựng sơ đồ lớp. Trong phạm vi báo cáo, ba use case nhỏ được lựa chọn và trình bày theo cách tiếp cận này.

\subsection{UC Phân quyền tài khoản }
Use case \hi{Phân quyền và quản lý tài khoản} nhằm hỗ trợ việc quản trị người dùng trong hệ thống quản lý bãi gửi xe, đảm bảo các chức năng được truy cập đúng theo vai trò và hạn chế thao tác ngoài thẩm quyền. Cụ thể, use case cho phép Admin thực hiện các nghiệp vụ tạo mới tài khoản, cập nhật thông tin tài khoản và xóa tài khoản khỏi hệ thống. Đồng thời, hệ thống gắn vai trò cho Admin để kiểm soát quyền truy cập vào các nhóm chức năng như vận hành cổng vào/ra, quản lý hủy vé tháng, cấu hình bảng giá và xem báo cáo. Trong khi đó, Attendance chỉ có chức năng vận hành cổng vào/ra, hỗ trợ đăng ký/ gia hạn vé tháng và gửi yêu cầu hủy vé tháng.



Dựa trên đặc tả nghiệp vụ và kiến trúc hệ thống, các \textbf{lớp dự kiến tham gia} trong use case bao gồm:








\section{Tổng kết các class cần xây dựng}
Dựa trên kết quả phân tích các lớp kỹ thuật, kết hợp với việc tái cấu trúc các Controller/Service chính và bổ sung các thành phần hạ tầng (Infrastructure), mô hình lớp tổng thể của hệ thống được tổng hợp cụ thể như sau:

\begin{longtable}{|C{1cm}|C{2cm}|L{3.0cm}|L{10cm}|}
\hline
\rowcolor{HeaderPink}
\textbf{STT} & \textbf{Loại} & \textbf{Tên class} & \textbf{Properties \& Methods (đầy đủ)} \\
\hline
\endfirsthead

\hline
\rowcolor{HeaderPink}
\textbf{STT} & \textbf{Loại} & \textbf{Tên class} & \textbf{Properties \& Methods (đầy đủ)} \\
\hline
\endhead

\hline
\endfoot

\hline
\endlastfoot

% ===== 1 =====
1& Boundary &
EntryUI &
\textbf{Properties:} \texttt{vehicleInfo}, \texttt{gateId}\newline
\textbf{Methods:} \texttt{requestCheckIn(vehicleInfo, gateId)}, \texttt{showTicketInfo(ticketInfo)}, \texttt{showFullMessage()}, \texttt{showError(message:String)} \\
\hline

% ===== 2 =====
2& Boundary &
ExitUI &
\textbf{Properties:} \texttt{ticketOrPlate}, \texttt{gateId}\newline
\textbf{Methods:} \texttt{requestCheckOut(ticketOrPlate, gateId)}, \texttt{showLostTicketForm()}, \texttt{showFeeInfo(feeAmount:double)}, \texttt{showError(message:String)} \\
\hline

% ===== 3 =====
3& Boundary &
MembershipUI &
\textbf{Properties:} \texttt{customerInfo}, \texttt{vehicleInfo}, \texttt{plan}\newline
\textbf{Methods:} \texttt{registerMonthlyTicket(customerInfo, vehicleInfo, plan)}, \texttt{showMembershipInfo(ticketInfo)}, \texttt{showError(message:String)} \\
\hline

% ===== 4 =====
4& Boundary &
UserManagementUI &
\textbf{Properties:} \texttt{info}, \texttt{userId}\newline
\textbf{Methods:} \texttt{createUserAccount(info)}, \texttt{updateUserAccount(info)}, \texttt{deleteUserAccount(userId)}, \texttt{showUserList()}, \texttt{showError(message:String)} \\
\hline

% ===== 5 =====
5& Boundary &
ReportUI &
\textbf{Properties:} \texttt{type:ReportType}, \texttt{start}, \texttt{end}, \texttt{data:ReportData}\newline
\textbf{Methods:} \texttt{selectReportType(type:ReportType)}, \texttt{inputDateRange(start,end)}, \texttt{displayReport(data:ReportData)}, \texttt{exportReport(format:String)} \\
\hline

% ===== 6 =====
6& Boundary/ Device &
GateDevice &
\textbf{Properties:} \texttt{gateId}\newline
\textbf{Methods:} \texttt{readPlate():String}, \texttt{readCardId():String}, \texttt{openGate(gateId)} \\
\hline

% ===== 7 =====
7 & Interface &
IPaymentGateway &
\textbf{Methods:} \texttt{requestPayment(amount:double, paymentInfo): PaymentResult} \\
\hline

% ===== 8 =====
8 & Boundary (adapter) &
PaymentGateway Adapter &
\textbf{Properties:} \texttt{config} \newline
\textbf{Methods:} \texttt{requestPayment(amount:double, paymentInfo): PaymentResult} \textit{(implements \texttt{IPaymentGateway})} \\
\hline

% ===== 9 =====
9 & Controller &
CheckInController &
\textbf{Properties:} \texttt{sessionRepo: IParkingSessionRepository}, \texttt{ticketRepo: ITicketRepository}\newline
\textbf{Methods:} \texttt{requestCheckIn(vehicleInfo, gateId)} \\
\hline

% ===== 10 =====
10 & Controller &
CheckOut Controller &
\textbf{Properties:} \texttt{sessionRepo: IParkingSessionRepository}, \texttt{feePolicy: PricePolicy}\newline
\textbf{Methods:} \texttt{requestCheckOut(ticketOrPlate, gateId)}, \texttt{processFoundSession(session:ParkingSession, gateId)}, \texttt{processLostTicket(vehicleInfo)} \\
\hline

% ===== 11 =====
11  & Controller &
Membership Controller &
\textbf{Properties:} \texttt{customerRepo: ICustomerRepository}, \texttt{monthlyTicketRepo: IMonthlyTicketRepository}\newline
\textbf{Methods:} \texttt{verifyMembership(vehicle:Vehicle): MembershipInfo}, \texttt{registerMonthlyTicket(customerInfo, vehicleInfo, plan)}, \texttt{findOrCreateCustomer(customerInfo): Customer} \\
\hline

% ===== 12 =====
12 & Controller &
Payment Controller &
\textbf{Properties:} \texttt{paymentGateway: IPaymentGateway}\newline
\textbf{Methods:} \texttt{processPayment(session:ParkingSession, amount:double, method:String)}, \texttt{processPaymentForMembership(customer:Customer, amount:double)} \\
\hline

% ===== 13 =====
13 & Controller &
UserAccount Controller &
\textbf{Properties:} \texttt{userRepo: IUserRepository}\newline
\textbf{Methods:} \texttt{createUserAccount(info)}, \texttt{updateUserAccount(info)}, \texttt{deleteUserAccount(userId)}, \texttt{getUserList(): List<UserAccount>} \\
\hline

% ===== 14 =====
14 & Controller &
SystemScheduler &
\textbf{Properties:} \texttt{monthlyTicketRepo: IMonthlyTicketRepository}\newline
\textbf{Methods:} \texttt{checkExpiredMonthlyTickets()}, \texttt{runDailyReportJob()} \\
\hline

% ===== 15 =====
15 & Controller &
ReportController &
\textbf{Properties:} \texttt{sessionRepo: IParkingSessionRepository}\newline
\textbf{Methods:} \texttt{requestReport(type:String, start:Date, end:Date)}, \texttt{exportReport(reportId:String, format:String)} \\
\hline

% ===== 16 =====
16 & Repository/ Helper &
JsonFileHelper &
\textbf{Methods:} \texttt{(static) readList<T>(filePath:String): List<T>}, \texttt{(static) writeList<T>(filePath:String, data:List<T>): bool} \\
\hline

% ===== 17 =====
17 & Interface &
IRepository<T> &
\textbf{Methods:} \texttt{getAll(): List<T>}, \texttt{getById(id:String): T}, \texttt{add(entity:T): void}, \texttt{update(entity:T): void}, \texttt{delete(id:String): void} \\
\hline

% ===== 18 =====
18 & Interface &
IParkingSession Repository &
\textbf{Methods:} \texttt{(CRUD IRepository<ParkingSession>)} + \texttt{findActiveByPlate(plate:String): List<ParkingSession>}, \texttt{findByTicketId(ticketId:String): ParkingSession}, \texttt{findByDateRange(start:Date, end:Date): List<ParkingSession>} \\
\hline

% ===== 19 =====
19 & Interface &
ICustomer Repository &
\textbf{Methods:} \texttt{(CRUD IRepository<Customer>)} + \texttt{findByPhone(phone:String): Customer} \\
\hline

% ===== 20 =====
20 & Interface &
IMonthlyTicket Repository &
\textbf{Methods:} \texttt{(CRUD IRepository<MonthlyTicket>)} + \texttt{findActiveByPlate(plate:String): MonthlyTicket}, \texttt{findExpiredTickets(date:Date): List<MonthlyTicket>} \\
\hline

% ===== 21 =====
21 & Interface &
ITicketRepository &
\textbf{Methods:} \texttt{(CRUD IRepository<Ticket>)} \\
\hline

% ===== 22 =====
22 & Interface &
IUserRepository &
\textbf{Methods:} \texttt{(CRUD IRepository<UserAccount>)} + \texttt{findByUsername(username:String): UserAccount} \\
\hline

% ===== 23 =====
23 & Repository &
JsonParking SessionRepo &
\textbf{Properties:} \texttt{filePath:String="sessions.json"}\newline
\textbf{Methods:} \texttt{implements IParkingSessionRepository} \texttt{(CRUD + findActiveByPlate, findByTicketId, findByDateRange)} \\
\hline

% ===== 24 =====
24 & Repository (impl) &
JsonCustomer Repo &
\textbf{Properties:} \texttt{filePath:String="customers.json"}\newline
\textbf{Methods:} \texttt{implements ICustomerRepository} \texttt{(CRUD + findByPhone)} \\
\hline

% ===== 25 =====
25 & Repository (impl) &
JsonMonthly TicketRepo &
\textbf{Properties:} \texttt{filePath:String="monthly\_tickets.json"}\newline
\textbf{Methods:} \texttt{implements IMonthlyTicketRepository} \texttt{(CRUD + findActiveByPlate, findExpiredTickets)} \\
\hline

% ===== 26 =====
26 & Repository (impl) &
JsonUserRepo &
\textbf{Properties:} \texttt{filePath:String="users.json"}\newline
\textbf{Methods:} \texttt{implements IUserRepository} \texttt{(CRUD + findByUsername)} \\
\hline

% ===== 27 =====
27 & Repository (impl) &
JsonTicketRepo &
\textbf{Properties:} \texttt{filePath:String="tickets.json"}\newline
\textbf{Methods:} \texttt{implements ITicketRepository} \texttt{(CRUD)} \\
\hline

% ===== 28 =====
28 & Entity &
ParkingLot &
\textbf{Properties:} \texttt{name:String}, \texttt{zones: List<ParkingZone>}\newline
\textbf{Methods:} \texttt{findZoneFor(vehicle:Vehicle, gateId:String): ParkingZone} \\
\hline

% ===== 29 =====
29 & Entity &
ParkingZone &
\textbf{Properties:} \texttt{zoneId:String}, \texttt{name:String}, \texttt{vehicleCategory:String}, \texttt{electricOnly:bool}, \texttt{capacity:int}, \texttt{activeSessions: List<ParkingSession>}, \texttt{pricePolicy: PricePolicy}\newline
\textbf{Methods:} \texttt{isFull(): bool}, \texttt{addSession(session:ParkingSession)}, \texttt{removeSession(session:ParkingSession)} \\
\hline

% ===== 30 =====
30 & Entity &
ParkingSession &
\textbf{Properties:} \texttt{sessionId:String}, \texttt{entryTime:DateTime}, \texttt{exitTime:DateTime}, \texttt{feeAmount:double}, \texttt{status:String}\newline
\textbf{Methods:} \texttt{setExitTime(time:DateTime)}, \texttt{close()}, \texttt{attachPayment(payment:Payment)} \\
\hline

% ===== 31 =====
31 & Entity &
Ticket &
\textbf{Properties:} \texttt{ticketId:String}, \texttt{issueTime:DateTime}, \texttt{gateId:String} \\
\hline

% ===== 32 =====
32 & Entity (abstract) &
Vehicle &
\textbf{Properties:} \texttt{licensePlate:String}\newline
\textbf{Methods:} \texttt{getFeeFactor(): double} \\
\hline

% ===== 33 =====
33 & Entity &
Car &
\textbf{Methods:} \texttt{getFeeFactor(): double} \\
\hline

% ===== 34 =====
34 & Entity &
ElectricCar &
\textbf{Methods:} \texttt{getFeeFactor(): double} \\
\hline

% ===== 35 =====
35 & Entity &
Motorbike &
\textbf{Methods:} \texttt{getFeeFactor(): double} \\
\hline

% ===== 36 =====
36 & Entity &
ElectricMotorbike &
\textbf{Methods:} \texttt{getFeeFactor(): double} \\
\hline

% ===== 37 =====
37 & Entity &
Bicycle &
\textbf{Methods:} \texttt{getFeeFactor(): double} \\
\hline


% ===== 39 =====
38 & Entity (abstract) &
PricePolicy &
\textbf{Properties:} \texttt{policyId:String}, \texttt{name:String}\newline
\textbf{Methods:} \texttt{calculateFee(session:ParkingSession): double} \\
\hline

% ===== 40 =====
39 & Entity &
ParkingFeePolicy &
\textbf{Methods:} \texttt{calculateFee(session:ParkingSession): double} \\
\hline

% ===== 41 =====
40 & Entity &
LostTicket FeePolicy &
\textbf{Methods:} \texttt{calculateFee(session:ParkingSession): double} \\
\hline

% ===== 42 =====
41 & Entity &
Payment &
\textbf{Properties:} \texttt{paymentId:String}, \texttt{amount:double}, \texttt{time:DateTime}, \texttt{method:String}, \texttt{status:String}\newline
\textbf{Methods:} \texttt{markCompleted()} \\
\hline

% ===== 43 =====
42 & Entity &
Customer &
\textbf{Properties:} \texttt{customerId:String}, \texttt{name:String}, \texttt{phone:String}\newline
\textbf{Methods:} \textit{(thuần dữ liệu trong diagram)} \\
\hline

% ===== 44 =====
43 & Entity &
MonthlyTicket &
\textbf{Properties:} \texttt{ticketId:String}, \texttt{startDate:Date}, \texttt{expiryDate:Date}, \texttt{status:String}\newline
\textbf{Methods:} \textit{(trong diagram hiện tại chưa liệt kê, có thể bổ sung nếu cần)} \\
\hline

% ===== 45 =====
44 & Entity &
MembershipPolicy &
\textbf{Properties:} \texttt{policyId:String}, \texttt{name:String}\newline
\textbf{Methods:} \texttt{calculateMembershipFee(plan:String, vehicleType:String): double} \\
\hline

% ===== 46 =====
45 & Entity (abstract) &
UserAccount &
\textbf{Properties:} \texttt{userId:String}, \texttt{username:String}, \texttt{passwordHash:String}, \texttt{status:String}\newline
\textbf{Methods:} \textit{(không nêu trong diagram)} \\
\hline

% ===== 47 =====
46 & Entity &
AdminAccount &
\textbf{Methods:} \texttt{canAccessEntry(): bool}, \texttt{canAccessExit(): bool}, \texttt{canAccessReports(): bool}, \texttt{canManageUsers(): bool} \\
\hline

% ===== 48 =====
47 & Entity &
AttendantAccount &
\textbf{Methods:} \texttt{canAccessEntry(): bool}, \texttt{canAccessExit(): bool}, \texttt{canAccessReports(): bool}, \texttt{canManageUsers(): bool} \\
\hline

% ===== 49 =====
48 & Entity (abstract) &
Report &
\textbf{Properties:} \texttt{startDate:Date}, \texttt{endDate:Date}, \texttt{generatedDate:DateTime}\newline
\textbf{Methods:} \texttt{generateData(): void}, \texttt{getExportData(): Object} \\
\hline

% ===== 50 =====
49 & Entity &
RevenueReport &
\textbf{Properties:} \texttt{totalRevenue:double}, \texttt{revenueByPaymentMethod:Map}\newline
\textbf{Methods:} \texttt{generateData()} \\
\hline

% ===== 51 =====
50 & Entity &
TrafficReport &
\textbf{Properties:} \texttt{totalVehicles:int}, \texttt{vehiclesByType:Map}\newline
\textbf{Methods:} \texttt{generateData()} \\
\hline

% ===== 52 =====
51 & Entity &
IncidentReport &
\textbf{Methods:} \texttt{generateData()} \\
\hline

% ===== 53 =====
52 & DTO/ Support &
MembershipInfo &
\textbf{Properties:} \texttt{hasMonthlyTicket: bool} \textit{(tối thiểu)}\\
\hline

% ===== 54 =====
53 & DTO/ Support &
PaymentResult &
\textbf{Properties:} \texttt{success: bool}, \texttt{message:String}, \texttt{transactionId:String} \textit{(nếu bạn dùng)} \\
\hline

% ===== 55 =====
54 & Enum/ Support &
ReportType &
\textbf{Values:} \texttt{REVENUE}, \texttt{TRAFFIC}, \texttt{INCIDENT} \\
\hline

% ===== 56 =====
55 & DTO/ Support &
ReportData &
\textbf{Properties:} \texttt{reportId:String}, \texttt{type:ReportType}, \texttt{payload:Object} \textit{(tối thiểu để UI hiển thị)} \\
\hline

56 & Infrastructure & UnitOfWork & \textbf{Methods:} \textit{runInTx(fn)} hoặc \texttt{begin() / commit() / rollback()}\\
\hline

57 & TimeProvider &
ReportData &
\textbf{Methods:} \texttt{nowDateTime(): DateTime, today(): Date} \\
\hline

58 & Infrastructure &SystemTimeProvider &
\textbf{Methods:} \texttt{nowDateTime(): DateTime, today(): Date} \\
\hline

59 & Infrastructure &MembershipValidator &
\textbf{Methods:} \texttt{isRenewPlanValid(months: int): bool, canRenew(ticket: MonthlyTicket, now: Date): bool, canCancel(ticket: MonthlyTicket, now: Date): bool} \\
\hline

60 & Infrastructure & LostTicketValidator &
\textbf{Methods:} \texttt{verifyOwnership(vehicleInfo, ownerDocs): bool} \\
\hline

\end{longtable}

\begin{tikzpicture}[x=0.00001pt]
  \draw (0,0) -- (2000000,0);
\end{tikzpicture}
\begin{figure}[H]
    \centering
    \includegraphics[width=1.0\linewidth]{Pic/class diam oop.drawio (1).png}
    \caption{Sơ đồ class của hệ thống}
    \label{fig:placeholder}
\end{figure}